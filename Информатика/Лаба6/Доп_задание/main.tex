\documentclass{article}
\usepackage[utf8]{inputenc}
\usepackage[russian]{babel}
\usepackage{graphicx}
\usepackage[letterpaper,top=1.5cm,bottom=2cm,left=3cm,right=3cm]{geometry}
\usepackage{tikz}
\usepackage{multicol}
\usepackage{color}
\usepackage{amssymb}
\usepackage{wrapfig}
\begin{document}
\thispagestyle{empty}
\begin{multicols*}{2} 
\hspace{10mm} \Large КНИГА I ПРЕДЛ.XXV.ТЕОРЕМА \qquad \normalsize \\ 
\begin{wrapfigure}{l}{0.15\textwidth}
    \includegraphics[width=0.15\textwidth]{pom.png}
\end{wrapfigure}
\begin{textit}
    
сли у двух треугольников две стороны
\tikz{
\draw[line width = 0.8 mm,blue] (0,0)  -- node[above] {\color{black}$A$} (0,0) -- (1.2,0) node[above] {\color{black}$B$} ;
}
и 
\tikz{
\draw[line width = 0.8 mm,red] (0,0)  -- node[above] {\color{black}$C$} (0,0) -- (1.2,0) node[above] {\color{black}$A$} ;
}
соответственно равны двум сторонам
\tikz{
\draw[blue] (0,0)  -- node[above] {\color{black}$D$} (0,0) -- (1.2,0) node[above] {\color{black}$E$} ;
}
и
\tikz{
\draw[red] (0,0)  -- node[above] {\color{black}$F$} (0,0) -- (1.2,0) node[above] {\color{black}$D$} ;
}
другого, но основания неравны, то угол над большим основанием
\tikz{
\draw[line width = 0.8 mm] (0,0)  -- node[above] {$B$} (0,0) -- (1.2,0) node[above] {$C$} ;
}
одного треугольника меньше угла под меньшим
\tikz{
\draw[orange] (0,0)  -- node[above] {\color{black}$E$} (0,0) -- (1.2,0) node[above] {\color{black}$F$} ;
}
другого.

\vspace{1cm}
\centering
\hspace*{-4em}\raisebox{-1em}{\begin{tikzpicture}
\fill[color = orange]  (0,0) node[above] {\tiny\color{black}$A$} -- (-0.24,-0.5) node[below]
{\tiny\color{black}$C$} -- (0.16,-0.5) node[below] {\tiny\color{black}$B$} --  (-0.24,-0.5) arc (-120:-60:0.4);
\end{tikzpicture}}
\end{textit}
\textbf{= , >} или \textbf{<}
\hspace*{-0em}\raisebox{-1.2em}{\begin{tikzpicture}
\fill[color = black]  (0,0) node[above] {\tiny\color{black}$D$} -- (-0.2,-0.5) node[below]
{\tiny\color{black}$F$} -- (0.08,-0.5) node[below] {\tiny\color{black}$E$} --(-0.2,-0.5) arc (-140:-45:0.2);;
\end{tikzpicture}}

\centering
\hspace*{-4em}\raisebox{-1em}{\begin{tikzpicture}
\fill[color = orange]  (0,0) node[above] {\tiny\color{black}$A$} -- (-0.24,-0.5) node[below]
{\tiny\color{black}$C$} -- (0.16,-0.5) node[below] {\tiny\color{black}$B$} --  (-0.24,-0.5) arc (-120:-60:0.4);
\end{tikzpicture}}
не равен
\raisebox{-1.2em}{\begin{tikzpicture}
\fill[color = black]  (0,0) node[above] {\tiny\color{black}$D$} -- (-0.2,-0.5) node[below]
{\tiny\color{black}$F$} -- (0.08,-0.5) node[below] {\tiny\color{black}$E$} --(-0.2,-0.5) arc (-140:-45:0.2);;
\end{tikzpicture}}
\textbf{,}

поскольку если
\raisebox{-1em}{\begin{tikzpicture}
\fill[color = orange]  (0,0) node[above] {\tiny\color{black}$A$} -- (-0.24,-0.5) node[below]
{\tiny\color{black}$C$} -- (0.16,-0.5) node[below] {\tiny\color{black}$B$} --  (-0.24,-0.5) arc (-120:-60:0.4);
\end{tikzpicture}}
\textbf{=}
\raisebox{-1.2em}{\begin{tikzpicture}
\fill[color = black]  (0,0) node[above] {\tiny\color{black}$D$} -- (-0.2,-0.5) node[below]
{\tiny\color{black}$F$} -- (0.08,-0.5) node[below] {\tiny\color{black}$E$} --(-0.2,-0.5) arc (-140:-45:0.2);;
\end{tikzpicture}}
\textbf{,} то

\tikz{
\draw[line width = 0.8 mm] (0,0)  -- node[above] {$C$} (0,0) -- (1.2,0) node[above] {$B$} ;
}
\textbf{=}
\tikz{
\draw[orange] (0,0)  -- node[above] {\color{black}$F$} (0,0) -- (1.2,0) node[above] {\color{black}$E$} ;
}
(пр. I.\scriptsize 4\normalsize ) \textbf{,}

что противоречит гипотезе;
\vspace{1 cm}

\raisebox{-1em}{\begin{tikzpicture}
\fill[color = orange]  (0,0) node[above] {\tiny\color{black}$A$} -- (-0.24,-0.5) node[below]
{\tiny\color{black}$C$} -- (0.16,-0.5) node[below] {\tiny\color{black}$B$} --  (-0.24,-0.5) arc (-120:-60:0.4);
\end{tikzpicture}}
не меньше
\raisebox{-1.2em}{\begin{tikzpicture}
\fill[color = black]  (0,0) node[above] {\tiny\color{black}$D$} -- (-0.2,-0.5) node[below]
{\tiny\color{black}$F$} -- (0.08,-0.5) node[below] {\tiny\color{black}$E$} --(-0.2,-0.5) arc (-140:-45:0.2);;
\end{tikzpicture}}
\textbf{,}

поскольку если
\raisebox{-1em}{\begin{tikzpicture}
\fill[color = orange]  (0,0) node[above] {\tiny\color{black}$A$} -- (-0.24,-0.5) node[below]
{\tiny\color{black}$C$} -- (0.16,-0.5) node[below] {\tiny\color{black}$B$} --  (-0.24,-0.5) arc (-120:-60:0.4);
\end{tikzpicture}}
\textbf{<}
\raisebox{-1.2em}{\begin{tikzpicture}
\fill[color = black]  (0,0) node[above] {\tiny\color{black}$D$} -- (-0.2,-0.5) node[below]
{\tiny\color{black}$F$} -- (0.08,-0.5) node[below] {\tiny\color{black}$E$} --(-0.2,-0.5) arc (-140:-45:0.2);;
\end{tikzpicture}}
\textbf{,}

то
\tikz{
\draw[line width = 0.8 mm] (0,0)  -- node[above] {$C$} (0,0) -- (1.2,0) node[above] {$B$} ;
}
\textbf{<}
\tikz{
\draw[orange] (0,0)  -- node[above] {\color{black}$F$} (0,0) -- (1.2,0) node[above] {\color{black}$E$} ;
}
(пр. I.\scriptsize 24\normalsize ) \textbf{,}

что противоречит гипотезе.

\(\therefore\)
\raisebox{-1em}{\begin{tikzpicture}
\fill[color = orange]  (0,0) node[above] {\tiny\color{black}$A$} -- (-0.24,-0.5) node[below]
{\tiny\color{black}$C$} -- (0.16,-0.5) node[below] {\tiny\color{black}$B$} --  (-0.24,-0.5) arc (-120:-60:0.4);
\end{tikzpicture}}
\textbf{>}
\raisebox{-1.1em}{\begin{tikzpicture}
\fill[color = black]  (0,0) node[above] {\tiny\color{black}$D$} -- (-0.2,-0.5) node[below]
{\tiny\color{black}$F$} -- (0.08,-0.5) node[below] {\tiny\color{black}$E$} --(-0.2,-0.5) arc (-140:-45:0.2);;
\end{tikzpicture}}
\textbf{. }
\textbf{ ~~~~~~~~~~~~~~~~~~~~~~~~~~~~~~~~~~~~~~~~~~~~~~~~~~~~~ч.т.д}\\
\raisebox{-1em}{\begin{tikzpicture}
\fill[color = orange]  (0,0) node[above] {\tiny\color{black}} -- (-0.24,-0.5) node[below]
{\tiny\color{black}} -- (0.16,-0.5) node[below] {\tiny\color{black}} --  (-0.24,-0.5) arc (-120:-60:0.4);
\draw  [very thick, blue](0,0)node[above, black] {$A$}  -- (1.7,-4.3)
node[below, black] {$B$}; 
\draw  [very thick, black](1.7,-4.3) --(-2,-5)
node[below left, black] {$C$};
\draw  [very thick, red](0,0) -- (-2,-5);
\end{tikzpicture}}

\raisebox{-1.2em}{\begin{tikzpicture}
\fill[color = black]  (0,0) node[above] {\tiny\color{black}} -- (-0.2,-0.5) node[below]
{\tiny\color{black}} -- (0.08,-0.5) node[below] {\tiny\color{black}} --(-0.2,-0.5) arc (-140:-45:0.2);
\draw[very thick, blue](0,0)node[above, black] {$E$}  -- (0.9,-4)
node[below, black] {$D$}; 
\draw[very thick, orange](0.9,-4) --(-2,-5)
node[below left, black] {$F$};
\draw[very thick, red](0,0) -- (-2,-5);
\end{tikzpicture}}
\end{multicols*}
\end{document}


















